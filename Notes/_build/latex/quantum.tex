% Generated by Sphinx.
\def\sphinxdocclass{report}
\documentclass[letterpaper,10pt,english]{sphinxmanual}
\usepackage[utf8]{inputenc}
\DeclareUnicodeCharacter{00A0}{\nobreakspace}
\usepackage{cmap}
\usepackage[T1]{fontenc}
\usepackage{babel}
\usepackage{times}
\usepackage[Bjarne]{fncychap}
\usepackage{longtable}
\usepackage{sphinx}
\usepackage{multirow}

\usepackage{dsfont}
\usepackage{slashed}
\usepackage{yfonts}
\usepackage{mathrsfs}
\def\degrees{^\circ}
\def\d{{\rm d}}

\def\sign{\mathop{\mathrm{sign}}}
\def\L{{\mathcal L}}
\def\H{{\mathcal H}}
\def\M{{\mathcal M}}
\def\matrix{}
\def\fslash#1{#1 \!\!\!/}
\def\F{{\bf F}}
\def\R{{\bf R}}
\def\J{{\bf J}}
\def\x{{\bf x}}
\def\y{{\bf y}}
\def\h{{\rm h}}
\def\a{{\rm a}}
\newcommand{\bfx}{\mbox{\boldmath $x$}}
\newcommand{\bfy}{\mbox{\boldmath $y$}}
\newcommand{\bfz}{\mbox{\boldmath $z$}}
\newcommand{\bfv}{\mbox{\boldmath $v$}}
\newcommand{\bfu}{\mbox{\boldmath $u$}}
\newcommand{\bfF}{\mbox{\boldmath $F$}}
\newcommand{\bfJ}{\mbox{\boldmath $J$}}
\newcommand{\bfU}{\mbox{\boldmath $U$}}
\newcommand{\bfY}{\mbox{\boldmath $Y$}}
\newcommand{\bfR}{\mbox{\boldmath $R$}}
\newcommand{\bfg}{\mbox{\boldmath $g$}}
\newcommand{\bfc}{\mbox{\boldmath $c$}}
\newcommand{\bfxi}{\mbox{\boldmath $\xi$}}

\newcommand{\bra}[1]{\left\langle #1\right|}
\newcommand{\ket}[1]{\left| #1\right\rangle}
\newcommand{\braket}[2]{\langle #1 \mid #2 \rangle}
\newcommand{\avg}[1]{\left< #1 \right>}


%\def\back{\!\!\!\!\!\!\!\!\!\!}
\def\back{}
\def\col#1#2{\left(\matrix{#1#2}\right)}
\def\row#1#2{\left(\matrix{#1#2}\right)}
\def\mat#1{\begin{pmatrix}#1\end{pmatrix}}
\def\matd#1#2{\left(\matrix{#1\back0\cr0\back#2}\right)}
\def\p#1#2{{\partial#1\over\partial#2}}
\def\cg#1#2#3#4#5#6{({#1},\,{#2},\,{#3},\,{#4}\,|\,{#5},\,{#6})}
\def\half{{\textstyle{1\over2}}}
\def\jsym#1#2#3#4#5#6{\left\{\matrix{
{#1}{#2}{#3}
{#4}{#5}{#6}
}\right\}}
\def\diag{\hbox{diag}}

\font\dsrom=dsrom10
\def\one{\hbox{\dsrom 1}}

\def\res{\mathop{\mathrm{Res}}}

\def\mathnot#1{\text{"$#1$"}}


%See Character Table for cmmib10:
%http://www.math.union.edu/~dpvc/jsmath/download/extra-fonts/cmmib10/cmmib10.html
\font\mib=cmmib10
\def\balpha{\hbox{\mib\char"0B}}
\def\bbeta{\hbox{\mib\char"0C}}
\def\bgamma{\hbox{\mib\char"0D}}
\def\bdelta{\hbox{\mib\char"0E}}
\def\bepsilon{\hbox{\mib\char"0F}}
\def\bzeta{\hbox{\mib\char"10}}
\def\boldeta{\hbox{\mib\char"11}}
\def\btheta{\hbox{\mib\char"12}}
\def\biota{\hbox{\mib\char"13}}
\def\bkappa{\hbox{\mib\char"14}}
\def\blambda{\hbox{\mib\char"15}}
\def\bmu{\hbox{\mib\char"16}}
\def\bnu{\hbox{\mib\char"17}}
\def\bxi{\hbox{\mib\char"18}}
\def\bpi{\hbox{\mib\char"19}}
\def\brho{\hbox{\mib\char"1A}}
\def\bsigma{\hbox{\mib\char"1B}}
\def\btau{\hbox{\mib\char"1C}}
\def\bupsilon{\hbox{\mib\char"1D}}
\def\bphi{\hbox{\mib\char"1E}}
\def\bchi{\hbox{\mib\char"1F}}
\def\bpsi{\hbox{\mib\char"20}}
\def\bomega{\hbox{\mib\char"21}}

\def\bvarepsilon{\hbox{\mib\char"22}}
\def\bvartheta{\hbox{\mib\char"23}}
\def\bvarpi{\hbox{\mib\char"24}}
\def\bvarrho{\hbox{\mib\char"25}}
\def\bvarphi{\hbox{\mib\char"27}}

%how to use:
%$$\alpha\balpha$$
%$$\beta\bbeta$$
%$$\gamma\bgamma$$
%$$\delta\bdelta$$
%$$\epsilon\bepsilon$$
%$$\zeta\bzeta$$
%$$\eta\boldeta$$
%$$\theta\btheta$$
%$$\iota\biota$$
%$$\kappa\bkappa$$
%$$\lambda\blambda$$
%$$\mu\bmu$$
%$$\nu\bnu$$
%$$\xi\bxi$$
%$$\pi\bpi$$
%$$\rho\brho$$
%$$\sigma\bsigma$$
%$$\tau\btau$$
%$$\upsilon\bupsilon$$
%$$\phi\bphi$$
%$$\chi\bchi$$
%$$\psi\bpsi$$
%$$\omega\bomega$$
%
%$$\varepsilon\bvarepsilon$$
%$$\vartheta\bvartheta$$
%$$\varpi\bvarpi$$
%$$\varrho\bvarrho$$
%$$\varphi\bvarphi$$

%small font
\font\mibsmall=cmmib7
\def\bsigmasmall{\hbox{\mibsmall\char"1B}}

\def\Tr{\hbox{Tr}\,}
\def\Arg{\hbox{Arg}}
\def\atan{\hbox{atan}}


\title{Quantum Notes}
\date{February 25, 2014}
\release{1.38}
\author{Lei Ma}
\newcommand{\sphinxlogo}{}
\renewcommand{\releasename}{Release}
\makeindex

\makeatletter
\def\PYG@reset{\let\PYG@it=\relax \let\PYG@bf=\relax%
    \let\PYG@ul=\relax \let\PYG@tc=\relax%
    \let\PYG@bc=\relax \let\PYG@ff=\relax}
\def\PYG@tok#1{\csname PYG@tok@#1\endcsname}
\def\PYG@toks#1+{\ifx\relax#1\empty\else%
    \PYG@tok{#1}\expandafter\PYG@toks\fi}
\def\PYG@do#1{\PYG@bc{\PYG@tc{\PYG@ul{%
    \PYG@it{\PYG@bf{\PYG@ff{#1}}}}}}}
\def\PYG#1#2{\PYG@reset\PYG@toks#1+\relax+\PYG@do{#2}}

\expandafter\def\csname PYG@tok@gd\endcsname{\def\PYG@tc##1{\textcolor[rgb]{0.63,0.00,0.00}{##1}}}
\expandafter\def\csname PYG@tok@gu\endcsname{\let\PYG@bf=\textbf\def\PYG@tc##1{\textcolor[rgb]{0.50,0.00,0.50}{##1}}}
\expandafter\def\csname PYG@tok@gt\endcsname{\def\PYG@tc##1{\textcolor[rgb]{0.00,0.27,0.87}{##1}}}
\expandafter\def\csname PYG@tok@gs\endcsname{\let\PYG@bf=\textbf}
\expandafter\def\csname PYG@tok@gr\endcsname{\def\PYG@tc##1{\textcolor[rgb]{1.00,0.00,0.00}{##1}}}
\expandafter\def\csname PYG@tok@cm\endcsname{\let\PYG@it=\textit\def\PYG@tc##1{\textcolor[rgb]{0.25,0.50,0.56}{##1}}}
\expandafter\def\csname PYG@tok@vg\endcsname{\def\PYG@tc##1{\textcolor[rgb]{0.73,0.38,0.84}{##1}}}
\expandafter\def\csname PYG@tok@m\endcsname{\def\PYG@tc##1{\textcolor[rgb]{0.13,0.50,0.31}{##1}}}
\expandafter\def\csname PYG@tok@mh\endcsname{\def\PYG@tc##1{\textcolor[rgb]{0.13,0.50,0.31}{##1}}}
\expandafter\def\csname PYG@tok@cs\endcsname{\def\PYG@tc##1{\textcolor[rgb]{0.25,0.50,0.56}{##1}}\def\PYG@bc##1{\setlength{\fboxsep}{0pt}\colorbox[rgb]{1.00,0.94,0.94}{\strut ##1}}}
\expandafter\def\csname PYG@tok@ge\endcsname{\let\PYG@it=\textit}
\expandafter\def\csname PYG@tok@vc\endcsname{\def\PYG@tc##1{\textcolor[rgb]{0.73,0.38,0.84}{##1}}}
\expandafter\def\csname PYG@tok@il\endcsname{\def\PYG@tc##1{\textcolor[rgb]{0.13,0.50,0.31}{##1}}}
\expandafter\def\csname PYG@tok@go\endcsname{\def\PYG@tc##1{\textcolor[rgb]{0.20,0.20,0.20}{##1}}}
\expandafter\def\csname PYG@tok@cp\endcsname{\def\PYG@tc##1{\textcolor[rgb]{0.00,0.44,0.13}{##1}}}
\expandafter\def\csname PYG@tok@gi\endcsname{\def\PYG@tc##1{\textcolor[rgb]{0.00,0.63,0.00}{##1}}}
\expandafter\def\csname PYG@tok@gh\endcsname{\let\PYG@bf=\textbf\def\PYG@tc##1{\textcolor[rgb]{0.00,0.00,0.50}{##1}}}
\expandafter\def\csname PYG@tok@ni\endcsname{\let\PYG@bf=\textbf\def\PYG@tc##1{\textcolor[rgb]{0.84,0.33,0.22}{##1}}}
\expandafter\def\csname PYG@tok@nl\endcsname{\let\PYG@bf=\textbf\def\PYG@tc##1{\textcolor[rgb]{0.00,0.13,0.44}{##1}}}
\expandafter\def\csname PYG@tok@nn\endcsname{\let\PYG@bf=\textbf\def\PYG@tc##1{\textcolor[rgb]{0.05,0.52,0.71}{##1}}}
\expandafter\def\csname PYG@tok@no\endcsname{\def\PYG@tc##1{\textcolor[rgb]{0.38,0.68,0.84}{##1}}}
\expandafter\def\csname PYG@tok@na\endcsname{\def\PYG@tc##1{\textcolor[rgb]{0.25,0.44,0.63}{##1}}}
\expandafter\def\csname PYG@tok@nb\endcsname{\def\PYG@tc##1{\textcolor[rgb]{0.00,0.44,0.13}{##1}}}
\expandafter\def\csname PYG@tok@nc\endcsname{\let\PYG@bf=\textbf\def\PYG@tc##1{\textcolor[rgb]{0.05,0.52,0.71}{##1}}}
\expandafter\def\csname PYG@tok@nd\endcsname{\let\PYG@bf=\textbf\def\PYG@tc##1{\textcolor[rgb]{0.33,0.33,0.33}{##1}}}
\expandafter\def\csname PYG@tok@ne\endcsname{\def\PYG@tc##1{\textcolor[rgb]{0.00,0.44,0.13}{##1}}}
\expandafter\def\csname PYG@tok@nf\endcsname{\def\PYG@tc##1{\textcolor[rgb]{0.02,0.16,0.49}{##1}}}
\expandafter\def\csname PYG@tok@si\endcsname{\let\PYG@it=\textit\def\PYG@tc##1{\textcolor[rgb]{0.44,0.63,0.82}{##1}}}
\expandafter\def\csname PYG@tok@s2\endcsname{\def\PYG@tc##1{\textcolor[rgb]{0.25,0.44,0.63}{##1}}}
\expandafter\def\csname PYG@tok@vi\endcsname{\def\PYG@tc##1{\textcolor[rgb]{0.73,0.38,0.84}{##1}}}
\expandafter\def\csname PYG@tok@nt\endcsname{\let\PYG@bf=\textbf\def\PYG@tc##1{\textcolor[rgb]{0.02,0.16,0.45}{##1}}}
\expandafter\def\csname PYG@tok@nv\endcsname{\def\PYG@tc##1{\textcolor[rgb]{0.73,0.38,0.84}{##1}}}
\expandafter\def\csname PYG@tok@s1\endcsname{\def\PYG@tc##1{\textcolor[rgb]{0.25,0.44,0.63}{##1}}}
\expandafter\def\csname PYG@tok@gp\endcsname{\let\PYG@bf=\textbf\def\PYG@tc##1{\textcolor[rgb]{0.78,0.36,0.04}{##1}}}
\expandafter\def\csname PYG@tok@sh\endcsname{\def\PYG@tc##1{\textcolor[rgb]{0.25,0.44,0.63}{##1}}}
\expandafter\def\csname PYG@tok@ow\endcsname{\let\PYG@bf=\textbf\def\PYG@tc##1{\textcolor[rgb]{0.00,0.44,0.13}{##1}}}
\expandafter\def\csname PYG@tok@sx\endcsname{\def\PYG@tc##1{\textcolor[rgb]{0.78,0.36,0.04}{##1}}}
\expandafter\def\csname PYG@tok@bp\endcsname{\def\PYG@tc##1{\textcolor[rgb]{0.00,0.44,0.13}{##1}}}
\expandafter\def\csname PYG@tok@c1\endcsname{\let\PYG@it=\textit\def\PYG@tc##1{\textcolor[rgb]{0.25,0.50,0.56}{##1}}}
\expandafter\def\csname PYG@tok@kc\endcsname{\let\PYG@bf=\textbf\def\PYG@tc##1{\textcolor[rgb]{0.00,0.44,0.13}{##1}}}
\expandafter\def\csname PYG@tok@c\endcsname{\let\PYG@it=\textit\def\PYG@tc##1{\textcolor[rgb]{0.25,0.50,0.56}{##1}}}
\expandafter\def\csname PYG@tok@mf\endcsname{\def\PYG@tc##1{\textcolor[rgb]{0.13,0.50,0.31}{##1}}}
\expandafter\def\csname PYG@tok@err\endcsname{\def\PYG@bc##1{\setlength{\fboxsep}{0pt}\fcolorbox[rgb]{1.00,0.00,0.00}{1,1,1}{\strut ##1}}}
\expandafter\def\csname PYG@tok@kd\endcsname{\let\PYG@bf=\textbf\def\PYG@tc##1{\textcolor[rgb]{0.00,0.44,0.13}{##1}}}
\expandafter\def\csname PYG@tok@ss\endcsname{\def\PYG@tc##1{\textcolor[rgb]{0.32,0.47,0.09}{##1}}}
\expandafter\def\csname PYG@tok@sr\endcsname{\def\PYG@tc##1{\textcolor[rgb]{0.14,0.33,0.53}{##1}}}
\expandafter\def\csname PYG@tok@mo\endcsname{\def\PYG@tc##1{\textcolor[rgb]{0.13,0.50,0.31}{##1}}}
\expandafter\def\csname PYG@tok@mi\endcsname{\def\PYG@tc##1{\textcolor[rgb]{0.13,0.50,0.31}{##1}}}
\expandafter\def\csname PYG@tok@kn\endcsname{\let\PYG@bf=\textbf\def\PYG@tc##1{\textcolor[rgb]{0.00,0.44,0.13}{##1}}}
\expandafter\def\csname PYG@tok@o\endcsname{\def\PYG@tc##1{\textcolor[rgb]{0.40,0.40,0.40}{##1}}}
\expandafter\def\csname PYG@tok@kr\endcsname{\let\PYG@bf=\textbf\def\PYG@tc##1{\textcolor[rgb]{0.00,0.44,0.13}{##1}}}
\expandafter\def\csname PYG@tok@s\endcsname{\def\PYG@tc##1{\textcolor[rgb]{0.25,0.44,0.63}{##1}}}
\expandafter\def\csname PYG@tok@kp\endcsname{\def\PYG@tc##1{\textcolor[rgb]{0.00,0.44,0.13}{##1}}}
\expandafter\def\csname PYG@tok@w\endcsname{\def\PYG@tc##1{\textcolor[rgb]{0.73,0.73,0.73}{##1}}}
\expandafter\def\csname PYG@tok@kt\endcsname{\def\PYG@tc##1{\textcolor[rgb]{0.56,0.13,0.00}{##1}}}
\expandafter\def\csname PYG@tok@sc\endcsname{\def\PYG@tc##1{\textcolor[rgb]{0.25,0.44,0.63}{##1}}}
\expandafter\def\csname PYG@tok@sb\endcsname{\def\PYG@tc##1{\textcolor[rgb]{0.25,0.44,0.63}{##1}}}
\expandafter\def\csname PYG@tok@k\endcsname{\let\PYG@bf=\textbf\def\PYG@tc##1{\textcolor[rgb]{0.00,0.44,0.13}{##1}}}
\expandafter\def\csname PYG@tok@se\endcsname{\let\PYG@bf=\textbf\def\PYG@tc##1{\textcolor[rgb]{0.25,0.44,0.63}{##1}}}
\expandafter\def\csname PYG@tok@sd\endcsname{\let\PYG@it=\textit\def\PYG@tc##1{\textcolor[rgb]{0.25,0.44,0.63}{##1}}}

\def\PYGZbs{\char`\\}
\def\PYGZus{\char`\_}
\def\PYGZob{\char`\{}
\def\PYGZcb{\char`\}}
\def\PYGZca{\char`\^}
\def\PYGZam{\char`\&}
\def\PYGZlt{\char`\<}
\def\PYGZgt{\char`\>}
\def\PYGZsh{\char`\#}
\def\PYGZpc{\char`\%}
\def\PYGZdl{\char`\$}
\def\PYGZhy{\char`\-}
\def\PYGZsq{\char`\'}
\def\PYGZdq{\char`\"}
\def\PYGZti{\char`\~}
% for compatibility with earlier versions
\def\PYGZat{@}
\def\PYGZlb{[}
\def\PYGZrb{]}
\makeatother

\begin{document}

\maketitle
\tableofcontents
\phantomsection\label{index::doc}


Some notes for quantum


\chapter{Introduction}
\label{index:introduction}\label{index:quantum-notes}
Some notes continued from the full theoretical physics notes are \href{http://cosmologytaskforce.github.io/PhysicsResearchSurvivalManual/}{here}.


\chapter{Table of Contents}
\label{index:table-of-contents}

\section{Vocabulary}
\label{vocabulary::doc}\label{vocabulary:vocabulary}
Vocabulary of physics, the fountain of research ideas.
\begin{enumerate}
\setcounter{enumi}{-1}
\item {} 
Fine Structure Constant

\end{enumerate}

$\alpha = \frac{k_\mathrm{e} e^2}{\hbar c} = \frac{1}{(4 \pi \varepsilon_0)} \frac{e^2}{\hbar c} = \frac{e^2 c \mu_0}{2 h}$

In electrostatic cgs units, $\alpha = \frac{e^2}{\hbar c}$.

In natural units, $\alpha = \frac{e^2}{4 \pi}$ .
\begin{enumerate}
\item {} 
Hydrogen Atom

\end{enumerate}

Potential $V(r) = -\frac{Z e^2}{4\pi \epsilon_0 r}$.

Energy levels: $E_{n} = -\left(\frac{Z^2 \mu e^4}{32 \pi^2\epsilon_0^2\hbar^2}\right)\frac{1}{n^2} = -\left(\frac{Z^2\hbar^2}{2\mu a_{\mu}^2}\right)\frac{1}{n^2} = \frac{\mu c^2Z^2\alpha^2}{2n^2}.$

Ground state of hydrogen atom $\psi_{100}(r)=\frac{1}{\sqrt{\pi}}\frac{1}{a^{3/2}} e^{-Z r/a}$.


\section{Approximation Methods}
\label{approx::doc}\label{approx:approximation-methods}

\subsection{Variational Method}
\label{approx:variational-method}

\subsubsection{Trial functions}
\label{approx:trial-functions}
Some of the calculable trial functions:
\begin{enumerate}
\item {} 
$\psi(x) = \cos\alpha x$, for $|\alpha x|<\pi/2$, otherwise 0.

\item {} 
$\psi(x) = \alpha^2 - x^2$, for $|x|<\alpha$, otherwise 0.

\item {} 
$\psi(x) = C \exp(-\alpha x^2/2)$.

\item {} 
$\psi(x) = C(\alpha - |x|)$, for $|x|<\alpha$, otherwise 0.

\item {} 
$\psi(x) = C\sin\alpha x$, for $|\alpha x|<\pi$, otherwise 0.

\end{enumerate}


\subsubsection{Procedure}
\label{approx:procedure}\begin{enumerate}
\item {} 
Pick a trial function.

\begin{notice}{note}{Note:}
How to pick a trial function? For ground state energy, we should pick a function that has the same property as the real ground state. This requires some understanding of the problem we are dealing with.

Things to consider:
\begin{enumerate}
\item {} 
The new problem is just a modification of a known solved problem. Then we can easily find out what really is different and interprete the new problem in terms of the old one.

\item {} 
If the Hamiltonian have definite parity, the ground state wave function should pick up some parity which is usually even to make it the lowest energy.

\item {} 
Continious function? A $C^\infty$ Hamiltonian can only have continious functions as solutions for a finite system.

\item {} 
Nodes deteremines the kinetic energy so check the nodes for ground state wave function.

\item {} 
Check the behivior of the wave function at different limits. In most cases, the Shrödinger equation can be reduced to something solvable at some limits.

\item {} 
\textbf{One more thing, the trial function should make the problem calculable.}

\end{enumerate}
\end{notice}

\end{enumerate}


\subsubsection{Why Not General Viriational Method}
\label{approx:why-not-general-viriational-method}
Why don't we just use a most general variational method to find out the ground state? Because we will eventually come back to the time-independent Shrödinger equation.

Suppose we have a functional form
\begin{gather}
\begin{split}E(\psi^*, \psi, \lambda) = \int dx \psi^* H \psi - \lambda \left( \int dx \psi^* \psi - 1 \right)\end{split}\notag\\\begin{split}\end{split}\notag
\end{gather}
The reason we have this Lagrange multiplier method is that the wave function should be normalized and this multiplier provides the degree of freedom. We would only get a wrong result if we don't include this DoF.

Variation of $\psi^*$,
\begin{gather}
\begin{split}\delta E = \int dx \delta \psi^* H \psi - \int dx \delta \psi^* \psi = 0\end{split}\notag\\\begin{split}\end{split}\notag
\end{gather}
Now what?
\begin{gather}
\begin{split}H \psi - \lambda \psi = 0\end{split}\notag\\\begin{split}\end{split}\notag
\end{gather}
Not helpful.


\subsection{Variational Method and Virial Theorem}
\label{approx:variational-method-and-virial-theorem}
For a potential $V(x)=b x^n$, we can prove that virial theorem is valid for ground state if we use Gaussian trial function $e^{- \alpha x^2/2}$.

A MMA proof is here.

Virial theorem is pretty interesting. It shares the same math with equipartition theorem.


\subsection{WKB}
\label{approx:wkb}
This is a semi-classical method. It is semi classical because we will use the classical momentum
\begin{gather}
\begin{split}\hbar k(x) = \sqrt{2m (E - V(x))}\end{split}\notag\\\begin{split}\end{split}\notag
\end{gather}
The following points are important for this method.
\begin{enumerate}
\setcounter{enumi}{-1}
\item {} 
WKB start from a classical estimation of wave number at a certain energy $E$ which is later quantified by the Bohr-Sommerfeld quantization rule.

\item {} 
Conservation law:
\begin{gather}
\begin{split}\frac{\partial}{\partial t}\rho + \nabla \cdot \vec j = 0\end{split}\notag\\\begin{split}\end{split}\notag
\end{gather}
where $\rho = \psi^* \psi$, $\vec j = -\frac{\hbar}{2 m i} \left( \psi \nabla \psi^* - \psi^* \nabla \psi \right)$. This can be derived from Shrödinger equation easily.

\item {} 
Phase:
Wave function is generally $A(x)\exp(\phi(x))$. However, $\phi(x)$ should be the area of the phase function starting from some initial point. For example in WKB, $k(x) = \phi'(x)$ and $\phi(x) = \int \phi'(x')d x' = \int k(x') d x'$.

Using this general wave function and conservation law we find out that $A(x) ~ \frac{1}{\sqrt{k(x)}}$. Then we can apply the two boundary conditions. However we will find two different wave functions given by two boundary conditions. Now we should connect them because $\psi(a) = \psi(b)$ exactly. By comparing the two wave functions we can find something like Bohr-Sommerfeld quantization rule.

\item {} 
Correction at bouldary:
However, this method requires that the potential varies slowly or equivalently the wave number varies slowly. Basicly we are just using the following approximation:
\begin{gather}
\begin{split}A'(x) = 0, k'(x) = 0\end{split}\notag\\\begin{split}\end{split}\notag
\end{gather}
For example when taking the derivative of wave function,
\begin{gather}
\begin{split}\psi'(x) = A'(x) e^{i\int \cdots} + A(x) k(x) e^{i\int \cdots} \approx A(x) k(x) e^{i\int \cdots}\end{split}\notag\\\begin{split}\end{split}\notag
\end{gather}
where we drop the term with $A'(x)$. That is to say
\begin{gather}
\begin{split}|A'|\ll |A k| \Rightarrow |k'| \ll k^2\end{split}\notag\\\begin{split}\end{split}\notag
\end{gather}
But at boundary where $E = V$, this is obviously not valid because $k=0$. So we need to fix this problem.

The solution is to use first order of the potential in a Taylor expansion. Then solve the problem exactly. Finally we connect regions that is far out from the boundary, need the boundary and between the boundary.

\end{enumerate}

If we can have a good boundary condition, then the energy spectrum given by WKB can be very good. Even we don't have a good boundary condtion, the excited states given by this method are always close to the exact ones.


\subsubsection{How does it work}
\label{approx:how-does-it-work}

\section{Symmetries in QM}
\label{symmetries::doc}\label{symmetries:symmetries-in-qm}

\subsection{The Invariant Quantity}
\label{symmetries:the-invariant-quantity}
First of all I want to know what is not changed or what is the invariant quantity in a transformation.

There are three kind of common transformations.
\begin{enumerate}
\item {} 
Time translation: move the system in time. In this sense time translation is just the time evolution operator or propagator.

\item {} 
Space translation: move the system in space.

\item {} 
Gauge transformation

\end{enumerate}

The invariance of them corresponds to:
\begin{enumerate}
\item {} 
Time translation invariance (T.T.I.) means the evolution of the system is not changing under time translations. \textbf{Hamiltonian is invariant.}

\item {} 
Space translation invariance (S.T.I.) means that the

\end{enumerate}


\subsubsection{Time Translation Symmetry}
\label{symmetries:time-translation-symmetry}
T.T.I. is generated by Hamiltonian which can be easily understood by looking into Shcrödinger equation.

The logic is to prove that Hamiltonian is time independent by using infinitesimal time translation approach. Given that Hamiltonian is time independent, we imediately know that time translation operator is just the propagator with the form
\begin{gather}
\begin{split}\hat T_{\Delta t} \equiv \hat U(\Delta t) = e^{-i \hat H \Delta t /\hbar}\end{split}\notag\\\begin{split}\end{split}\notag
\end{gather}
All other conclusions come from the fact that Hamiltonian is a constant of motion.

\begin{notice}{hint}{Hint:}
Starting from Schrödinger equation,
\begin{gather}
\begin{split}i\hbar \frac{\ket{\psi(t+\Delta t)} - \ket{\psi(t)}}{\Delta t} = H(t)\psi(t)\end{split}\notag\\\begin{split}\end{split}\notag
\end{gather}
Then we get the state after a evolution of time $\Delta t$,
\begin{gather}
\begin{split}\ket{\psi(t+\Delta t)} = \left( \hat I - i\frac{\Delta t \hat H(t)}{\hbar} \right) \ket{\psi(t)}\end{split}\notag\\\begin{split}\end{split}\notag
\end{gather}
Time translation symmetry means the state evolution in the same time interval $\Delta t$ no matter when to start the evolution. Mathematically,
\begin{gather}
\begin{split}\ket{\psi(t_1 +\Delta t)} = \left( \hat I - i\frac{\Delta t \hat H(t_1)}{\hbar} \right) \ket{\psi(t_1)}\end{split}\notag\\\begin{split}\end{split}\notag
\end{gather}
should get the same final state if we start from some other time $t_2$,
\begin{gather}
\begin{split}\ket{\psi(t_2 +\Delta t)} = \left( \hat I - i\frac{\Delta t \hat H(t_2)}{\hbar} \right) \ket{\psi(t_2)}\end{split}\notag\\\begin{split}\end{split}\notag
\end{gather}
That means the two Hamiltonian should be the same. Now we reach the conclusion that Hamiltonian is time independent.
\end{notice}

\begin{notice}{hint}{Hint:}
Ehrenfest theorem tells us that time independent Hamiltonian is a constant of motion.
\begin{gather}
\begin{split}\frac{d}{dt}\avg{H} = \frac{1}{i\hbar}\avg{[\hat H, \hat H]} + \avg{\frac{\partial}{\partial t} H } = 0\end{split}\notag\\\begin{split}\end{split}\notag
\end{gather}\end{notice}

\begin{notice}{important}{Important:}
For an isolated system, T.T.I. should always be satisfied because there is nothing more else to change the system but to leave the system with energy conserved.

My concern is if we don't have an Hamiltonian for $T\mathrm d S$, we can't actually says this because of what the second law of thermodynamics tells us.
\end{notice}


\subsubsection{Space Translation Symmetry}
\label{symmetries:space-translation-symmetry}
S.T.I. is generated by canonical momentum. This is not so obvious as time translation. To prove this we need to understand what space translation really means.

Space translation means we change the position of the system by some spatial distance $a$. In math this means a transformation from $\ket{x}$ to $\ket{x+a}$ where the plus sign is by definition. We invent this space translation operator,
\begin{gather}
\begin{split}\hat T_a  \ket{x} = \ket{x + a} .\end{split}\notag\\\begin{split}\end{split}\notag
\end{gather}
Next we can obtain the result of space translation operator applied to state in position basis
\begin{gather}
\begin{split}\bra{x}\hat T_a \ket{\psi} = (\bra{x}\hat T_a^\dagger) \ket{\psi} = \braket{x-a}{\psi} = \psi(x-a)\end{split}\notag\\\begin{split}\end{split}\notag
\end{gather}
where we used the relation
\begin{gather}
\begin{split}(\bra{x} \hat T_a^\dagger) =  \bra{x}\hat T_a^{-1} = \bra{x} \hat T _ {-a} = \ket{x-a}\end{split}\notag\\\begin{split}\end{split}\notag
\end{gather}
which of course is because the normalization of coordinate basis tells us that space translation operator is unitary.


\bigskip\hrule{}\bigskip



\bigskip\hrule{}\bigskip

\begin{figure}[htbp]
\centering
\href{http://creativecommons.org/licenses/by-nc-sa/3.0/us/}{\includegraphics{cc_byncsa.png}}\end{figure}


\bigskip\hrule{}\bigskip


This open source project is hosted on GitHub: \href{https://github.com/emptymalei/quantum}{quantum} .

\href{https://raw.github.com/emptymalei/quantum/master/Notes/\_build/latex/quantum.pdf}{Latest PDF here}.



\renewcommand{\indexname}{Index}
\printindex
\end{document}
